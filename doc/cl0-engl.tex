% Projekt  : APPLY - A Practicable And Portable Lisp Implementation
%            ------------------------------------------------------
%            CL_0 Sprachbeschreibung
% 
% $Revision: 1.1 $
% $Id: cl0-engl.tex,v 1.1 1993/09/03 15:01:38 wg Exp $

\documentstyle[11pt,twoside,apply,eapply,listing]{article}

%% RCS Time is GMT = (MET - 1) !
%% $Id: cl0-engl.tex,v 1.1 1993/09/03 15:01:38 wg Exp $


\def\RCSdate $#1 19#2/#3/#4 #5 ${#4.#3.#2}
\def\RCSrevision $#1 #2 ${#2}

\def\clos0{{\sc CLOS$_0$}}
\def\CLOS{{\sc CLOS}}
\newcommand{\option}[1]{\code{#1}}
\newcommand{\cliccnote}[1]{%
  \bigskip
  \begin{quote}
  {\bf Note:}\\
  {\it #1}
  \end{quote}
  \bigskip
}

\author{\wg, \uh, \hk}
\title{\cl0\ and \clos0: The Language Definition}
\issuer{Christian-Albrechts-Universit\"at zu Kiel}
\institution{CAU}
\task{Definition effizienter Teilsprachen}
\documentid{APPLY/CAU/II.2/2}
\status{Draft}
\created{09.08.93}
\revised{\RCSdate $Date: 1993/09/03 15:01:38 $}
\synopsis{This document includes the definition of the languages \cl0\
and \clos0\ as strict subsets of \CL\ and \CLOS.}

% man mu"s es trennen k"onnen.
\def\CL{{\sc Common\-Lisp}}
\def\cl0{{\sc Common\-Lisp}$_0$}

\def\clm{{\sc CL}$_{macro}$}
\def\.{\discretionary{}{}{}}

\begin{document}

\coverpages

%%------------------------------------------------------------------------------
\section{Introduction}

The program development process for \Lisp\ applications usually takes
place within an interactive development and runtime system. It
benefits from the possibilities to incrementally add and redefine
arbitrary parts of the program and to interactively test and debug
even incomplete program parts. Since many of the tools used to
facilitate this process have been expressed in \Lisp\ itself, it is
easy to build an application specific development environment.
Unfortunately, many of these tools actually have
become part of the language \CL\ itself and, as a
consequence, the application specific program development environment
often becomes part of the application program. Hence, many 
applications inherently include the ability to dynamically change their own 
program text, i.e.\ are {\em reflective}.

One of the major goals of the \APPLY\ project is to develop the techniques
needed for generating efficient stand alone applications which easily can 
be integrated with and into standard software environments. One of the 
techniques we use in order to achieve this goal is called {\em complete 
compilation} of programs or modules. It uses the complete set of 
information about the whole program or module in order to perform global 
and interprocedural program analysis and optimization. 

In order to prove those optimizations correct we have to ensure that
the program is not reflective, i.e.\ that it does not include the
ability to change its own program text. In general, this is impossible
for the full \CL\ language. Hence, this document defines a suitable
subset, essentially excluding the reflective parts of \CL. We call
this subset \cl0. It is a very large and strict subset of the \CL\ as
it is defined in \cite{Steele84} and \cite{Steele90}.

% Code, der vom Compiler zur "Ubersetzungszeit evaluiert werden mu"s,
% z.~B.\ w"ahrend der Ausf"uhrung von Makroexpansionsfunktionen oder zur
% Initialisierung von Konstanten, darf nicht auf die
% "Ubersetzungszeitumgebung zugreifen. Das hei"st, da"s z.~B.\ die
% Variable {\tt *package*} nicht explizit gesetzt werden darf, sondern
% da"s dazu nur das Makro {\tt in-package} verwendet werden darf. Durch
% diese Einschr"ankung werden Inkompatibilit"aten, die aus
% unterschiedlichen Ladereihenfolgen entstehen, ausgeschlossen.

% Die Verwendung von CLOS wird in entsprechender Weise eingeschr"ankt:
% Alle Klassen, generischen Funktionen und Methoden m"ussen schon zur
% "Ubersetzungszeit bekannt sein und d"urfen nicht ver"andert werden. Das
% wird zum einen dadurch erreicht, da"s {\tt eval} nicht verwendet
% werden darf und zum anderen werden Funktionen, die generische
% Funktionen zur Laufzeit erzeugen oder ver"andern (z.~B.\ {\tt
% add-method} oder {\tt ensure-generic-function}) aus dem Sprachumfang
% von \cl0\ entfernt.

%-------------------------------------------------------------------------------

%%------------------------------------------------------------------------------
\subsection{How to read this document}

This document has to be read carefully in conjunction with the
original \CL\ language definition, i.e.\ a copy of \cite{Steele84} or
\cite{Steele90}. We will define \cl0\ by stating the differences and 
changes with respect to the \CL\ definition instead of listing the whole 
language. The second edition of Guy Steele's book
(\cite{Steele90}) contains a complete copy of the first one, and
additionally contains

\begin{itemize}
\item corrections to the first edition which concern errors and inconsistencies,
\item clarifications formulated by the ANSI standardization committee X3J13,
\item extensions proposed by X3J13, and
\item changes proposed by X3J13.
\end{itemize}

The ANSI standardization committee X3J13 has been established in order
to define an ANSI standard for the language \CL. Up to now, this
standardization process is not yet finished. Therefore, the proposals
of X3J13 stated in \cite{Steele90} have to be taken as part of an
ongoing discussion about the language, which at the end might result
in a language definition for ANSI \CL\ even is incompatible with
\cite{Steele90}. Therefore, at present the \cl0\ definition is based
on \cite{Steele84}, together with the corrections stated in
\cite{Steele90}, i.e.\ on what is commonly referred to as CLtL1.

We will mention extensions and changes proposed by X3J13 in so far as
we are able to state whether or not they are {\em acceptable} for
future versions of \cl0. Proposals for changes or extensions to parts
of \CL\ which are not intended to belong to \cl0 will be called {\em
unuseful} throughout this document. In no sense this is meant to value
the proposals' quality. However, many constructs of the development
environment are unuseful as parts of a subset of the language which is
defined for the sake of generating stand alone applications.

Although this document is intended to be a language definition, it
will include some remarks stating differences between \cl0\ or \clos0\
and the language which is supported by the current release of our
Common Lisp to C Compiler CLiCC. We added these remarks in order to
make this document usable as part of the CLiCC documentation.

%%------------------------------------------------------------------------------
\subsection{A strict subset of \CL}

\cl0\ is intended to be a strict and very large subset of \CL. By {\em
  strict} we mean, that every well defined \cl0\ program is also a
well defined and semantically equivalent \CL\ program.

The restrictions we defined are different in nature. Some functions
like {\tt eval} or {\tt symbol-function} are simply not part of the
language \cl0, others are less defined. Some \CL\ functions like {\tt
  load} are defined to be compile time operations, and hence are
restricted to be used at top level of the program with constant
arguments only. There are both syntactical and semantical
restrictions, and some of the latter aren't even decidable. It might
depend on some additional invariants to hold, whether or not a program
actually is a well defined \cl0\ program. As an example, the
programmer has to ensure that the functional argument {\it form} in
{\tt (apply {\it form} ...)} neither evaluates to a symbol denoting a
function nor to a list starting with {\tt lambda}. Otherwise, the
\cl0\ program would cause an error, whereas the \CL\ program
eventually runs successfully.  Note, however, that this situation does
not destroy the property of \cl0\ beeing a strict subset of \CL.

\cl0\ is intended to be a very large subset of \CL. It is not our
major goal to define a well-structured new lisp-like language.
Instead we define \cl0\ as large as the approach of complete
compilation reasonably allows, in order to simplify the migration
process for real world \CL\ applications.

There are some practical advantages which result from \cl0\ beeing a
strict subset of \CL, and which we shall mention here: The programmer
may use a conventional Common Lisp system in order to develop, debug,
and test \cl0\ applications. Moreover, the strict subset property
allows to incrementally migrate \CL\ programs into \cl0, using the
development system as a runtime and test environment for the stepwise
resulting programs. We refer to \cite{APPLY/CAU/IV/2} for details on
migration.

%%------------------------------------------------------------------------------
\subsection{Notation}

As in \cite{Steele90} we will use the {\tt (setf f)} notation in order
to refer to the function which {\em updates} the {\em place} {\tt f}
results in. 

In \CL, lists which start with the symbol {\tt lambda} are often 
implicitely {\em coerced} to functions. We will call those lists {\em 
lambda-lists} throughout this document. Lambda-lists will not be coerced 
to functions in \cl0. Use {\tt \#'(lambda ...)} instead of {\tt '(lambda 
...)}. A similar restriction holds for symbols which actually are names of 
functions. We will often call them {\em function names}.

%%------------------------------------------------------------------------------
\section{\cl0}

As mentioned earlier, we will not define \cl0\ by completely listing
the whole language. Instead we will follow the original language
definition of \CL\ \cite{Steele84} chapter by chapter, stating only
restrictions and differences of \cl0\ with respect to \CL.
Therefore, it will be necessary to read a copy of \cite{Steele84} or
\cite{Steele90} at the same time in order to obtain the complete
definition.

The same strategy will be used in order to define \clos0. However, the
\clos0\ definition is subject of a separate chapter (\ref{clos0def})
of this document.

Some of the restrictions defined below are of syntactical nature. Some
of the \CL\ language constructs, for example, are restricted to be
used as top level forms, only. Please note, that in the following we
assume macro calls of the original program to be sufficiently expanded
in order to make those constructs visible.

The restrictions defined for \cl0\ roughly fall into four classes:

The language constructs which are only useful during the program
development process, like {\tt trace}, {\tt break} or {\tt compile-file},
are completely dropped from the language. We recommend the use of a
conventional \CL\ system for the development, debugging and test of
\cl0\ applications. Hence, the dropped functionality is part of the
development environment, but it needs not to be part of the
application itself.

We completely disallow {\em reflection}. Functions which dynamically
change the program code, like {\tt eval} and {\tt (setf
symbol-function)} are not part of \cl0. Unlike \CL, \cl0\ does not
implicitely {\em coerce} symbols or lists starting with {\tt lambda}
to functions. Hence, functions like {\tt apply} or {\tt funcall} are
restricted to be undefined, if the functional argument actually is a
symbol or a list. Multiple definitions of functions and/or macros are
not allowed.

In order to generate small application programs it is useful to be
able to statically determine the set of potentially called functions
within a program. Practically, this would be impossible if there were
nontrivial calls to {\tt symbol-function}. A very simple and in some
sense necessary method in order to avoid these complications is to
avoid {\tt symbol-function} at all. We additionally benefit from this
restriction, because for many functions, actually for all those which
are not explicitely used as functional objects, it now becomes
possible to statically determine all function calls.  This will enable
us to use classical interprocedural program analysis and optimization
techniques which rely on the knowledge of all calling contexts.

In order to enable separate compilation of Lisp modules, we restricted
the use of the \CL\ package operations to really express
modularization. The complex functionality of the \CL\ package system
seems to be useful only to avoid name conflicts while reloading or
recompiling program parts. And this, again, facilitates the program
development process, but needs not necessarily to be part of the
application itself.

\cliccnote{The current release of CLiCC does not support the complete
set of functions as they are defined to be part of \cl0. Moreover,
some functions, like {\tt format}, are not implemented completely.
We will document this in future versions. At present, please have a
look at the runtime system source code, i.e.\ the file {\tt
format.lisp} for the function {\tt format}.

CLiCC will support a foreign function interface to C in one
of the future releases which will not be part of \cl0\ as a strict
subset of \CL, and hence will be documented separately.
}

Note, that the titles of the following sections are equal to those
used in \cite{Steele84}.

%%------------------------------------------------------------------------------
\subsection{Introduction}

No changes.

%%------------------------------------------------------------------------------
\subsection{Data Types}

The \CL\ data type {\tt function} is restricted to contain ''real''
functions only, i.e.\ functions which are the result of evaluating the
special form {\tt function}. Function names, i.e.\ symbols which name
a function, and lambda-lists are not functions and will not be
implicitely coerced to functions. The range of system functions
which deal with functional arguments, like {\tt apply} or {\tt
mapcar}, is restricted to the \cl0\ data type {\tt function}. They are
listed in sections \ref{control}, \ref{sequences}, and
\ref{lists}. This restriction has already been proposed by X3J13.

\cliccnote{In the released version, CLiCC does not support bignums,
rational and complex numbers, and bit vectors. Moreover, only some of
the trigonometric functions are supported.}

X3J13 proposed to add the data types {\tt base-character} und {\tt
  extended-character} instead of the type {\tt extended-character},
which is acceptable. The same holds for the proposed extensions to
stream types.

%%------------------------------------------------------------------------------
\subsection{Scope and Extent}

No changes.

%%------------------------------------------------------------------------------
\subsection{Type Specifiers}
\label{type}

Type specifiers are classified to be part of the program text in \cl0. 
Therefore, we disallow type specifiers to be evaluated at runtime.
This would be the interpretation of data as code at runtime, hence {\em 
reflection}.

Type specifiers may occur as part of declarations, where they are 
completely known at compile time. \cl0 does not restrict type declarations.

In \CL, type specifiers are also allowed to be the result of argument
forms of functions like {\tt coerce}, {\tt typep}, {\tt subtypep},
{\tt make-sequence}, {\tt concatenate}, {\tt map}, {\tt merge}, {\tt
make-array}, {\tt adjust-array}, and {\tt open}. Consequently, we have
to restrict their type specifier arguments:

%Die Bedeutung der Typspezifikatoren, die den Typ der Parameter und
%Resultate von Funktionen beschreiben, ist nicht brauchbar,
%so wie sie in \cite{Steele84} beschrieben ist.

\begin{itemize}
\item {\tt coerce}, {\tt typep}, and {\tt subtypep} are defined to be 
macros. The type specifier arguments are required to be constants.
\item {\tt make-sequence}, {\tt concatenate}, {\tt map}, {\tt merge}, {\tt 
make-array}, {\tt adjust-array}, and {\tt open} are restricted to be 
defined only on special symbols and lists which denote known types. For 
further details see the sections \ref{sequences}, \ref{arrays}, 
\ref{strings} und \ref{io}.
\item {\tt type-of} is not part of \cl0. It evaluates type specifiers at 
runtime and hence has become obsolete in \cl0.
\end{itemize}

Note, that the first item above implies, that {\tt coerce}, {\tt typep}, 
and {\tt subtypep} cannot be used directly as functional objects in \cl0. 
But, of course, functions like {\tt \#'(lambda (x) (coerce x '(vector 
float)))} are allowed. 

Since {\tt type-of} is not part of \cl0, forms like {\tt (eq (type-of
  x) 'integer)} have to be expressed like {\tt (integerp x)}

The restrictions above imply that in \cl0\ type specifiers, which are 
defined using {\tt deftype} forms, can be expanded at 
compile time. Hence, {\tt deftype} has become syntactic sugar in \cl0.

The functions {\tt upgraded-\.array-\.element-\.type} and {\tt
upgraded-\.complex-\.part-\.type} proposed by X3J13 are acceptable
with similar restrictions. {\tt coerce}'ing a symbol to a function
explicitely, as proposed by X3J13, is just another notation
for {\tt symbol-function}, and hence is unuseful.

X3J13 proposed to handle type specifiers uniformly, no matter whether they 
occur as part of a declaration or as part of a method discriminator, and to 
add {\tt eql} type specifiers. Both proposals are acceptable. Moreover, the 
proposed change of the semantics of the {\tt function} type specifier is 
acceptable. It would allow type inference to actually use the information 
contained in type declarations, which, in turn, would improve the results 
of type inference for \CL\ and \cl0\ programs dramatically. We hope this to 
be part of the ANSI \CL\ specification and will then adopt it for future 
versions of \cl0.

%%------------------------------------------------------------------------------
\subsection{Program Structure}
\label{progstructure}

Calls of the \CL\ macros {\tt defun}, {\tt defvar}, {\tt
  def\-para\-meter}, {\tt def\-constant}, {\tt def\-type}, {\tt
  def\-setf}, {\tt define-\.setf-\.method}, {\tt
  define-\.modify-\.macro}, {\tt defmacro}, and {\tt defstruct} as
well as the \clos0\ macro calls to {\tt defclass}, {\tt defgeneric},
and {\tt defmethod} are called to be {\em defining forms} in \cl0.
They are restricted only to occur as {\em extended top level forms},
which we define as follows:

\cliccnote{At present, CLiCC will only accept {\tt (progn ...)} forms
as extended top level forms. Adding the others is part of our TODO list.}

Definition:
\label{erwTLF}

\begin{enumerate}

\item Every top level form is an {\em extended top level form}.
\item If a call of the special forms {\tt progn}, {\tt
let}, {\tt let*}, {\tt labels}, {\tt flet}, {\tt macro-let}, {\tt
eval-when}, or {\tt progv} is an {\em extended top level form}, then so are 
the forms occuring within the body. 
\end{enumerate}

% Die von X3J13 vorgeschlagenen Special Forms {\tt generic-flet}, {\tt
% generic-labels}, {\tt symbol-macrolet}, {\tt with-added-methods} und
% {\tt locally} werden in die Liste der eben genannten Special Forms
% aufgenommen werden.
% --> CLOS

X3J13 proposed to add the special forms {\tt symbol-macrolet} and
{\tt locally}. This is acceptable. Moreover, they could be added to the list of 
special forms mentioned in the above definition.

Note, that the above restrictions on the occurance of defining forms allow 
defining forms to occur within non empty lexical environments. The X3J13 
proposal to allow defining forms to occur in arbitrary contexts is 
unuseful. The reason is, that it has to remain decidable whether or not, and 
when, a definition actually takes place. Moreover, \cl0\ does not allow any 
redefinition:

Global Macros and global functions share the same namespace, in which multiple 
definitions are rejected. Note, that this restriction also concerns 
implicitely defined global functions, like those defined by {\tt 
defstruct}, and generic functions.

Likewise, constants and global variables, i.e.\ those names defined by {\tt 
defconstant} and {\tt defvar}, resp., also share a common namespace. 
Multiple {\tt defvar}'s are allowed, but constants may not be redefined to 
be global variables, and vice versa. Multiple definitions of the same 
constant are rejected.

The fact, that redefinitions are rejected in general, implies, of course, 
that predefined functions of the {\tt "LISP"} package may not be redefined, 
as proposed by X3J13.  The same holds for predefined constants of 
the {\tt "LISP"} package.

A restricted version of {\tt eval-when} is part of \cl0. The forms which 
have to be evaluated at compile time are restricted to be part of the 
language \clm\ which slightly differs from \cl0\ and is defined in section 
\ref{clm}. 

\cliccnote{CLiCC does not support {\tt eval-when} at present. This is
part of our TODO list.}

X3J13 proposed to add {\tt (setf f)} forms as legal function names. This is 
acceptable. The proposed changes of {\tt eval-when} are acceptable, as well 
as the proposal to allow arbitrary symbols to denote {\em keywords} in 
formal parameter lists.

%%------------------------------------------------------------------------------
\subsection{Predicates}
\label{predicates}

The operators {\tt typep} and {\tt subtypep} are defined to be macros
in \cl0. A macro call {\tt (typep {\it object type}\/)} is completely
expanded at compile time into an appropriate runtime system function
call.  E.g., {\tt (typep x 'symbol)} will be expanded into {\tt
  (symbolp x)}.  Since type specifiers are restricted to be constant
in \cl0\ (section \ref{type}), calls to {\tt subtypep} can be replaced
by their results.  {\tt (subtypep } {\em type1 type2} {\tt)} will be
expanded at compile time into a call of {\tt values}. In order to give
an example, {\tt (subtypep 'cons 'list)} will expand into {\tt (values
  t t)}. This very rigorous simplification will change in future
versions of \cl0\ with respect to class arguments of {\tt subtypep}.

\cliccnote{At present, {\tt subtypep} is not supported by CLiCC.}

The changes to {\tt typep} and {\tt subtypep} proposed by X3J13 are 
acceptable.

The predicate {\tt functionp} is restricted to be undefined on symbols
and lists. The reason is to preserve the strict subset property, i.e.\
{\tt functionp} may not be defined to yield a different result than
the \CL\ {\tt functionp} eventually would do if applied to a lambda
list or function name.  Note, that this restriction to {\tt functionp}
enlarges the set of non well defined \cl0\ programs. This is the price
we have to pay for preserving the strict subset property. However, the
function {\tt functionp} as proposed by X3J13, is acceptable and
could be part of \cl0\ without any restrictions.

\cliccnote{Hence, {\tt (functionp 'A)} should signal an error, which
is not the case in our present CLiCC release. Fixing this bug is part
of our TODO list.}

%%------------------------------------------------------------------------------
\subsection{Control Structure}
\label{control}

The functions {\tt symbol-function} and {\tt (setf symbol-function)} are 
not part of \cl0, nor are {\tt fboundp} and {\tt fmakunbound}. The reason 
for eliminating {\tt (setf symbol-function)} needs no further explanation. 
We eliminated {\tt symbol-function} as well in order to enable dead code 
elimination, and to keep the set of eventually called functions small. In 
\cl0, we know every function which eventually is called, since it has to be 
mentioned explicitely within the program text, either in a function call or 
in the form {\tt \#'f}.

The function {\tt special-form-p} is not part of \cl0.

The {\tt defsetf} and {\tt define-setf-method} macro bodies are completely 
evaluated at compile time. The functions {\tt get-setf-method} and {\tt
get-setf-method-multiple-value} are useless at runtime of a program and 
therefore do not belong to \cl0. They will be part of \clm.

Symbols and lambda-lists are not allowed as functional arguments to
the functions {\tt funcall}, {\tt apply}, {\tt mapcar}, {\tt maplist},
{\tt mapc}, {\tt mapl}, {\tt mapcan}, and {\tt map}. This is one of
the major restrictions in \cl0, and the reason once more is the
elimination of reflection: For any form {\it x}, the form {\tt
(funcall (list 'lambda () {\it x}))} is equivalent to {\tt (eval {\it
x})}.  Instead of lambda-lists or quoted symbols the programmer has to
use the forms {\tt \#'(lambda (}\ldots{\tt) \ldots\ )} or {\tt \#'f},
resp.  X3J13 already proposed to eliminate the automatic coercion of
lambda-lists to functions, which is acceptable. In \cl0, the
coercion of function names to functions is eliminated, too.

X3J13 proposed to add the function {\tt fdefinition}, which
essentially is the same as {\tt symbol-function}, and therefore is
unuseful.

As proposed already by X3J13, the special form {\tt compiler-let} is not 
part of \cl0. Adding the special form {\tt symbol-macrolet} would be 
acceptable.

%%------------------------------------------------------------------------------
\subsection{Macros}
\label{macro}

Like in \CL, macro expansion is completely performed at compile time. The 
language \clm\, as defined in section \ref{clm}, can be used to 
formulate expressions which have to be evaluated at compile time. It 
slightly differs from \cl0, but neither \clm\ is a subset of \cl0 nor vice 
versa. Note, that the result of the macro expansion has to fit into \cl0, 
whereas the forms that compute this result, have to be part of \clm.

The functions {\tt macro-function} and {\tt (setf macro-function)} do not 
belong to \cl0. Hence, the only way to define a global macro is to use the 
defining form {\tt defmacro}. A macro has to be defined before its first 
use. The redefinition of macros is not allowed, and a macro must not 
redefine a function already defined.

The functions {\tt macroexpand} and {\tt macroexpand-1} do not belong to 
\cl0, but they do belong to \clm.

\cliccnote{At present, {\tt macroexpand} and {\tt macroexpand-1} are
not part of \clm. This is part of our TODO list.}

The variable {\tt *macroexpand-hook*} is not part of \cl0.

Within this chapter, the X3J13 proposals mention the additional macros
and functions {\tt destruc\-turing-\.bind}, {\tt
  define-\.compiler-\.macro}, {\tt compiler-\.macro-\.function}, {\tt
  compiler-\.macro\-expand} {\tt compiler-\.macro\-expand-1}, {\tt
  variable-\.infor\-mation}, {\tt function-\.information}, {\tt
  decla\-ration-\.infor\-mation}, {\tt augment-\.environment}, {\tt
  define-\.declaration}, {\tt parse-\.macro}, and {\tt enclose}. They
are unuseful as parts of \cl0\ because it only makes sense to use them
during macro expansion. However, as parts of \clm\ they are
acceptable.

%%------------------------------------------------------------------------------
\subsection{Declarations}

The operator {\tt proclaim} is defined to be a macro in \cl0\, which only 
may occur as an extended top level form (see page \pageref{erwTLF}). Its 
arguments are evaluated at compile time. The macro {\tt declaim}, as it is 
proposed by X3J13, is acceptable. But it has to be restricted to only occur as 
extended top level form.

X3J13 proposed changes with respect to the semantics of type declarations, 
which are acceptable. They probably will improve the results of type 
inference for Lisp programs.
 
%%------------------------------------------------------------------------------
\subsection{Symbols}

No changes. 

X3J13 proposed changes with respect to {\tt gensym} and {\tt
*gensym-counter*}, which both are acceptable.


%%------------------------------------------------------------------------------
\subsection{Packages}
\label{packages}

In \cl0, packages may be used to express {\em modules}. A \cl0\ program then
consists of one or more modules. In \CL, on the other hand, packages are 
used to define separate name spaces for symbols, and the package operations 
altogether are a powerful tool to maintain these name spaces. At a first 
glance, these two views of packages have nothing in common. The reason why 
packages can be used to express modules is in some sense historical: The 
syntactical representation of a \CL\ program actually is a list, consisting 
of a lot of symbols. Since the reader has to be used to read a program
text like 
any other piece of data, symbols denoting program identifiers are 
modularized into different packages. 

Unfortunately, the \CL\ package operations are too powerful to express
a real modularization. The programmer is able to break every module
border. The possibility of using "::" package qualifiers or of
exporting symbols from different packages, are examples.

The good news is, however, that not every package operation at runtime
necessarily has to have an effect to the process of reading the
program text
itself, i.e.\ to the compile time environment. Therefore, package
operations within a program fall into two classes. Those which do not
occur at top level of the program, and hence are not evaluated at
compile time, may be used without any restriction. The restrictions
which have to hold for the top level or extended top level package
operations of a \cl0\ program, and which are defined below, are
intended to let packages express real modules in the sense of \EL\ or
Modula-2.

%
% Packages sollen in \cl0\ verwendet werden, um die Module eines Programms
% zu beschreiben. Ein Modul ist {\em ein} zusammenh"angendes
% Programmst"uck; es kann in der konkreten Realisierung auf eine oder
% mehrere Dateien verteilt sein. Dem APPLY-Modul-Compiler mu"s es
% erm"oglicht werden, alle Teile des Programms, die zu einem Modul
% geh"oren, in einem St"uck zu lesen und zu "ubersetzen.
%

A module starts with a call of {\tt in-package}. The only legal
package top level forms within the subsequent module text are calls of
{\tt export}, {\tt import}, {\tt shadow}, {\tt shadowing-import}, and
{\tt use-package} with arguments which are evaluated at compile time.
A top level package operation must not operate on a package other than
the current one. Therefore the optional package argument of the above
forms, if supplied, has to denote the current package. Function calls
of {\tt make-package}, {\tt unexport}, and {\tt unuse-package} as well
as write accesses to the global variable {\tt *package*} must
not be used as top level forms. Moreover, none of these forms
are allowed within any form of the
program which has to be evaluated at compile time. This restriction
will remain even if the use of global variables will be allowed in
\clm\ in future versions of \cl0. 

\cliccnote{At present, neither {\tt shadow} nor {\tt shadowing-import}
are supported as legal package top level forms. This is part of our
TODO list. It will be fixed as soon as the module compilation feature
gets released, which, however, is an inofficial part of the current
release already {\tt :-)}}

{\tt (in-package ...)} top level forms may occur repeatedly within one
module, unless the package argument denotes a different package.  This
is useful since \cl0\ modules may be separated into different files.

A {\tt (load {\it filename})} top level form loading a file which
starts with an {\tt (in-package ...)} form, changing into a new and
different package, serves as a module import. The {\em package}{\tt
:}{\em name} notation may be used to get access to the exported
identifiers of the imported module. Of course, {\tt import} or {\tt
use-package} may be used to enable this access without any package
qualification.  However, the {\em package}{\tt ::}{\em name} notation
is not allowed in \cl0.

A module $A$ {\em depends on} a module $B$ if one or more identifiers
of $B$ are referenced within the program text of $A$. This {\em module
dependency} relation has to form a DAG (directed acyclic graph) in
\cl0. This restriction ensures that all of the necessary information
about the exported functionality of $B$ is known while analysing $A$.
Hence, the separate compilation of modules is possible, starting with
the leaf modules of the module dependency graph of the program.

\cliccnote{The DAG property of the module dependency graph will not be
checked by CLiCC in it's current release. This is part of our TODO list.}

Within one module, global definitions are restricted to only use names
which have the current package as home package. One module must not change
the global definitions of another one:

Example:
%
{\listing \parskip0.8ex

(in-package "a")
  (export 'f)
  (defun f (x) (* x x))

(in-package "b")
  (export 'g)
  (defun g (x) (+ x x))

(in-package "c")
  (use-package "a")
  (f 3)
  (defvar f 4)     ; prohibited !
  (b:g 3)
  (defstruct b:g ) ; prohibited !
}

The changes to the package system proposed by X3J13 are acceptable.
The {\tt defpackage} top level form improves the possibility of using
packages as modules.

Like {\tt load}, the functions {\tt require} and {\tt provide} are
restricted to the extended top level of the program. {\tt require}
behaves like {\tt load}, but like in \CL\ it uses the variable {\tt
*modules*} in order to avoid reloading of modules.

%%------------------------------------------------------------------------------
\subsection{Numbers}

No changes.

%%------------------------------------------------------------------------------
\subsection{Characters}

No changes.

%%------------------------------------------------------------------------------
\subsection{Sequences}
\label{sequences}

Only the symbols {\tt list}, {\tt vector}, {\tt simple-vector}, {\tt 
string}, {\tt simple-string}, {\tt bit-vector}, or {\tt simple-bit-vector} 
are allowed to be result type specifications for the functions {\tt 
make-sequence}, {\tt concatenate}, {\tt map}, and {\tt merge}

The following functions do not accept function names or lambda-lists as 
functional argument:

{\tt map}, {\tt some}, {\tt every}, {\tt notany}, {\tt notevery}, {\tt
reduce}, {\tt remove}, {\tt remove-\.if}, {\tt remove-\.if-\.not},
{\tt delete}, {\tt delete-\.if}, {\tt delete-\.if-\.not}, {\tt
remove-\.dup\-licates}, {\tt delete-\.dup\-licates}, {\tt
sub\-stitute}, {\tt sub\-sti\-tute-\.if}, {\tt
sub\-sti\-tute-\.if-not}, {\tt nsub\-stitute}, {\tt
nsub\-sti\-tute-if}, {\tt nsub\-stitute-\.if-\.not}, {\tt find}, {\tt
find-\.if}, {\tt find-\.if-\.not}, {\tt position}, {\tt position-if},
{\tt position-\.if-\.not}, {\tt count}, {\tt count-\.if}, {\tt
count-\.if-\.not}, {\tt mismatch}, {\tt search}, {\tt sort}, {\tt
stable-sort}, {\tt merge}.

%%------------------------------------------------------------------------------
\subsection{Lists}
\label{lists}

The following functions do not accept function names or lambda-lists as 
functional argument:

{\tt tree-equal}, {\tt push-new}, {\tt subst}, {\tt subst-\.if}, {\tt
subst-\.if-\.not}, {\tt nsubst}, {\tt nsubst-\.if}, {\tt
nsubst-\.if-\.not}, {\tt sublis}, {\tt nsublis}, {\tt member}, {\tt
member-\.if}, {\tt member-\.if-\.not}, {\tt adjoin}, {\tt union}, {\tt
nunion}, {\tt inter\-section}, {\tt ninter\-section}, {\tt
set-\.difference}, {\tt nset-\.difference}, {\tt set-exclusive-or},
{\tt nset-exclusive-or}, {\tt subsetp}, {\tt assoc}, {\tt assoc-if},
{\tt assoc-\.if-\.not}, {\tt rassoc}, {\tt rassoc-\.if}, {\tt
rassoc-\.if-not}.

%%------------------------------------------------------------------------------
\subsection{Hash Tables}

The {\tt :test} argument of {\tt make-hash-table} has to be a functional 
object. Function names or lambda-lists are not allowed.

%%------------------------------------------------------------------------------
\subsection{Arrays}
\label{arrays}

The functions {\tt make-array} and {\tt adjust-array} are undefined on
{\tt :element-type} arguments other than the symbols {\tt t}, {\tt
float}, {\tt short-float}, {\tt single-float}, {\tt long-\.float}, {\tt
double-\.float}, {\tt fixnum}, {\tt bit}, {\tt character}, {\tt
standard-char}, {\tt string-char} or lists of the form {\tt
(signed-byte} {\em s} {\tt)} or {\tt (unsigned-byte} {\em s} {\tt)}.

As proposed by X3J13, the value of {\tt array-dimension-limit} has to be of 
type {\tt fixnum} in \cl0. The same holds for {\tt array-total-size-limit}.

%%------------------------------------------------------------------------------
\subsection{Strings}
\label{strings}

No changes.

%%------------------------------------------------------------------------------
\subsection{Structures}

Redefinitions of structures are not allowed. Name clashes between 
implicitely generated structure functions and any other global function or 
macro name will cause an error. Name clashes between the structure name and 
the name of any other type are rejected as well.

%%------------------------------------------------------------------------------
\subsection{The Evaluator}

\cl0 does not contain the functions {\tt eval}, {\tt evalhook}, {\tt 
applyhook} and {\tt constantp}, nor does it contain the variables {\tt 
*evalhook*} and {\tt *applyhook*}. 

Since \cl0\ is intended to be a language rather than an interactive 
programming system, the variables {\tt +}, {\tt ++}, {\tt +++}, {\tt -}, 
{\tt *}, {\tt **}, {\tt ***}, {\tt /}, {\tt //}, and {\tt ///} are not 
supported. They only make sense in order to control the {\tt 
read-eval-print} loop of an interactive \CL\ system.

%%------------------------------------------------------------------------------
\subsection{Streams}

The variables {\tt *debug-io*} and {\tt *trace-io*} are not part of \cl0.

%%------------------------------------------------------------------------------
\subsection{Input/Output}
\label{io}

One of the most dangerous language constructs in \CL\ is {\tt (eval 
(read))}. Note, that the input of a form like {\tt (defun f ...)} might 
even change the running program. The dispatching macros `{\tt \#.}' und 
`{\tt \#,}' exactly have this effect, i.e.\ evaluating the enclosed form 
while reading it. Therefore they do not belong to \cl0.

%%------------------------------------------------------------------------------
\subsection{File System Interface}
\label{file}

The {\tt :element-type} of the function {\tt open} has to be one of the 
symbols {\tt string-char}, {\tt character}, {\tt unsigned-byte}, {\tt 
signed-byte}, or {\tt bit}.

The operator {\tt load} is defined to be a macro, restricted to the 
extended top level of the program. The arguments are evaluated at compile 
time. {\tt (load {\it file}\/)} is substitutet by the content of {\it
  file}, or stands for a {\em module import},
if the situation is as
described in section \ref{packages}.

%%------------------------------------------------------------------------------
\subsection{Errors}

\cl0\ does not know anything about break loops or interactive error 
handling. The function {\tt cerror} will ask the user whether or not to 
continue the program. {\tt break} behaves like {\tt cerror}. {\tt error} 
aborts the program without any chance to continue.

The variable {\tt *break-on-warnings*} is not part of \cl0.

Since \cl0\ is a strict subset of \CL, any existing interactive \CL\ system 
will serve as a more or less comfortable debugging and test environment for 
\cl0\ programs. 

%%------------------------------------------------------------------------------
\subsection{Miscellaneous Features}

The functions {\tt compile}, {\tt compile-file}, {\tt disassemble},
and {\tt documentation} do not belong to \cl0.

{\tt trace}, {\tt untrace}, {\tt step}, {\tt time},
{\tt describe}, {\tt inspect}, {\tt room}, {\tt ed}, {\tt dribble},
{\tt apropos}, and {\tt apropos-list} are definitely part of the 
development environment of \CL\ systems, and therefore, are not part of 
the language \cl0.

%%------------------------------------------------------------------------------
\section{\clm}
\label{clm}

Like \cl0, \clm\ is a subset of \CL. But neither \clm\ is a subset of \cl0\ 
nor vice versa. Expressions of a \cl0\ program, which have to be evaluated 
at compile time, like macro bodies, have to belong to \clm. Expressions, 
which eventually are evaluated at compile time and at runtime, like
bodies of functions which are used to compute macro expansion results, have 
to satisfy the restrictions defined for both \cl0\ and \clm.

The definition of \clm\ is not yet finished. One of the open questions at 
present is whether or not we should allow global variables to be 
interpreted at compile time. We are aware of applications which use global 
variables for macro expanders to communicate with each others. But there 
are problems with the distinction between compile time and runtime values 
of those variables. They may vary if the compiler does not evaluate the 
whole program at compile time already. The problems are wellknown, and many 
programmers have had the same problems while writing portable \CL\ code.

The functions {\tt get-setf-method}, {\tt macroexpand}, and {\tt 
macroexpand-1} are part of \clm, but they do not belong to \cl0, since 
accessing the compile time environment makes no sense at runtime.

\cliccnote{At present, \clm\ is a subset of \cl0\ because the
additional features are not yet implemented.}

\clm\ does not include any package operations. Note, that e.g. macro 
expansion might result in package operations, but must not perform any 
package manipulation. Every package operation evaluated at compile
time has to occur explicitely as a legal package top level operation,
after sufficiently expanding top level macro calls, though.

%%% Ein weiterer Grund, die Sprache \clm\ einzuschr"anken, ist darin
%%% zu sehen, da"s ein \cl0-Compiler \clm-Ausdr"ucke zur "Ubersetzungszeit
%%% {\em evaluieren} mu"s und es m"oglich sein soll, diesen Compiler in
%%% \cl0\ zu schreiben.  Daraus ergibt sich die Notwendigkeit, in \cl0\
%%% einen Interpretierer f"ur \clm\ zu schreiben. Um dessen Umfang
%%% m"oglichst klein zu halten, kann es sinnvoll sein, den Sprachumfang
%%% von \clm\ bzgl.\ \cl0\ einzuschr"anken.

\CL\ does not specify anything about how a reasonable compile time
environment for evaluating forms at compile time looks like. For good
reasons, it is not defined whether or not {\em compiling} a definition
also means {\em evaluating} that definition within the compile time
environment, although many \CL\ systems will do so. Consequently,
nothing is defined for the compile time environment for \cl0.

% Loop: Ein Makro-Paket, da"s in \cl0\ beschreibbar ist. Allerdings mu"s
% \clm\ dazu m'achtig genug sein.

% Pretty-Printing: Nicht sehr relevant f"ur Applikationen. Falls es
% implementiert wird, dann mu's folgende Einschr"ankung vorgenommen
% werden: Der Teilstring ~/<Function name>/ zum Aufruf einer Funktion
% aus Format-Strings heraus ist nicht Bestandteil von \cl0\. Dies ist
% zumindest dann notwendig, wenn der Format-String dynamisch erzeugt
% wird. Die Einschr"ankung ist notwendig, damit zur "Ubersetzungszeit
% alle erreichbaren Funktionen bekannt sind.

% Conditions: Zur "Ubersetzungszeit sollte bekannt sein, welche
% Condition-Handler installiert sind, damit besser optimiert werden
% kann. 
% Beispiel: 
% (defun fak (x)
% ...
% (* ... ...))

% (fak (read))

% darf man ein spezielles read einsetzten, da"s nur Numbers einliest ?
% Oder kann * eine Condition ausl"osen, so da"s auch not-Numbers
% verarbeitet werden k"onnen ?

\section{\clos0}
\label{clos0def}

% Projekt  : APPLY - A Practicable And Portable Lisp Implementation
%            ------------------------------------------------------
%            CLOS_0 Sprachbeschreibung fuer cl0-engl.tex
% 
% $Revision: 1.1 $
% $Id: clos-0.tex,v 1.1 1993/09/03 15:01:38 wg Exp $

Like \cl0, which essentially is defined in order to exclude {\em
reflection} from \CL, \clos0\ will be defined as a subset of \CLOS,
which essentially excludes unrestricted uses of reflection.  The
reason again is, that we want to ensure the quality of {\em global}
program analysis and optimization for \cl0\ application programs which
use object oriented programming.

But, unlike reflection in \CL, in \CLOS\ the notion of reflection has
various aspects. First of all, the distinction between program and
data is not as simple and wellknown for the object system as for the
non object oriented kernel of \CL. Instances are data, of course. But
should classes, i.e.\ types, be program or data? Should classes be
instances of metaclasses?  At present, there are many different
answers to questions like this. The intention of \clos0\ is to call
those operations, which work on classes or generic functions as data
objects, {\em reflective}.  Hence, they are excluded from the
language. On the other hand, metaobject programming, at least in a
restricted manner, seems to be very useful even for application
programming. At least some parts of the metaobject protocol should be
supported by \clos0, whereas other parts, like class redefinition,
definitely belong to the development environment.  Although some
reflection should be part of \clos0, at present it is not. We
have to be very careful in improving the present definition in this
aspect, because we have to ensure both, the strict subset property,
and the quality of static program analysis.

The strategy we will use to improve \clos0\ in order to include
metaobject programming will be to keep those operations apart from
runtime, like runtime macro expansion has been removed from  the
modern \Lisp\ language versions. 

The definition on \clos0\ included here has to be taken as a
first attempt in order to define a suitable subset of \CLOS. At
present, the restrictions are very rigorous in order to achieve our
goals: There is no meta object protocol. Classes, generic functions
and methods are program, not data. There is no possibility to define
or redefine classes, methods, or generic functions at runtime,
nor to add methods to generic functions. Changing an
instance's class is impossible. There are objects representing classes,
but they have no class. It is an error to access a class's
class. Moreover, \clos0\ ensures class objects only to represent
classes which are statically defined within the program text.

%%% Internally they are represented as instances of the one and only
%%% invisible and illegal metaclass {\tt standard-class}.

Again, not every useful \CL\ $+$ \CLOS\ program belongs and will
belong to the subset defined within this document. But we hope that it
will be rich enough to express the most requirements of \Lisp\
application programs.

%%------------------------------------------------------------------------------
\subsection{Programmer Interface Concepts}
\label{concepts}

As before, we will follow the section headings of the original
specification, i.e.\ the chapter 28 of \cite{Steele90}, in order to
define \clos0\ by stating changes and differences with respect to the
original.  Section \ref{functions} then will summarize both the
functionality that is supported by \clos0\, together with the
restrictions defined, and those functions, macros, or generic
functions which are not.

Restrictions are defined both for the {\em metaobject protocol} and
the {\em functional interface}, which \CLOS\ provides in order to define
classes, generic functions, and methods at runtime. Therefore, in \clos0\ 
the only legal method to create classes, generic functions or methods is to 
use {\tt defclass}, {\tt defgeneric}, or {\tt defmethod}, which are {\em 
defining forms} and hence are restricted to occur as extended toplevel 
forms only. (cf.\ section \ref{progstructure})

The generic functions which are classified to be part of the
metaobject protocol, are:

\begin{itemize}
\item[$\bullet$] {\tt shared-initialize}
\item[$\bullet$] {\tt update-instance-for-different-class}
\item[$\bullet$] {\tt update-instance-for-redefined-class}
\item[$\bullet$] {\tt compute-applicable-methods}
\item[$\bullet$] {\tt ensure-generic-function}
\item[$\bullet$] {\tt compute-effective-method}
\item[$\bullet$] {\tt allocate-instance}
\item[$\bullet$] {\tt make-instances-obsolete}
\end{itemize}

They do not belong to \clos0. 
As a consequence, many functions and generic functions which remain in
\clos0, like {\tt initialize-instance},
actually do behave as this is defined for the standard
situation, but their behavior is not customizable using method
definitions for generic functions of the metaobject protocol. 

We classify {\tt add-method}, {\tt remove-method}, {\tt generic-function}, and calls to 
{\em make-instance} with metaclasses
as class arguments to belong to the functional interface. \clos0\ does not 
support them. Moreover, \clos0\ does not support metaclasses at all.
Thus, there is no way to create instances of metaclasses, like anonymous classes or generic 
functions, nor to dynamically change or enlarge the set of methods which 
belong to a generic function.

The lists above are intended to give a first impression of the 
nature of 
the restrictions defined within this document. Hence, they are not complete, 
i.e.\ there are more important restrictions, and more functions, like {\tt 
change-class}, are excluded, although they are neither part of the metaobject 
protocol nor do belong to the functional interface.
We refer to the subsequent sections for details.

%%------------------------------------------------------------------------------
\subsubsection{Classes}
\label{classes}

The classes {\tt standard-class}, {\tt standard-generic-function}, and {\tt 
standard-method} are metaclasses in \CLOS. Therefore, they do not belong to \clos0. 

Unlike in \CLOS, {\tt find-class} is restricted to be defined on constant
symbols which name user defined classes, i.e.\ a classes which are defined 
using {\tt defclass}.

The function {\tt make-instance} is not generic. It is restricted to be 
defined only on class objects, which, in turn, only can be accessed using 
{\tt find-class}. Hence, \clos0\ ensures class objects only to represent 
classes which are defined using the extended toplevel form {\tt 
defclass} and which, therefore, are statically known from the program text. 
In order to avoid unnecassary syntactical transformations, {\tt 
(make-instance '<{\it name}\/> ...)} internally is transformed to {\tt 
(make-instance (find-class '<{\it name}\/>) ...)}, and hence, is legal in 
\clos0.

The {\tt defclass} macro does not support the class option {\tt 
:metaclass}.

Moreover, the {\tt :allocation} slot option must not override the {\tt 
:allocation} option of an inherited slot. 

The standard methods defined for {\tt slot-unbound} and {\tt 
slot-missing} will signal and uncontinuable error and hence will abort the 
program execution. Note, that, according to the current definition of 
\clos0, there is no way to define methods for {\tt slot-unbound} and {\tt 
slot-missing} because they would have to be specialized over metaclasses. 
This is true for {\tt no-applicable-method} and {\tt no-next-method} as 
well (cf. section \ref{methodselection}). Hence, an implementation can choose to 
call simple error functions instead of these generic functions.

%%------------------------------------------------------------------------------
\subsubsection{Inheritance}
\label{inheritance}

The {\tt :allocation} slot option is restricted not to override specified {\tt 
:allocation} options of inherited slots. Thus, unlike in \CLOS, shared 
slots cannot be shadowed in \clos0.

%%------------------------------------------------------------------------------
\subsubsection{Integrating Types and Classes}
\label{classtypes}

The functions {\tt typep}, {\tt type-of}, and {\tt subtypep} are restricted 
according to the sections \ref{predicates} and \ref{type} of the 
\cl0\ part of this document. As 
mentioned there already, these very rigorous restrictions will be weakened 
with respect to classes in future versions of \cl0\ + \clos0.

The metaclasses {\tt built-in-class} and {\tt structure-class} do not 
belong to \clos0.

%%------------------------------------------------------------------------------
\subsubsection{Determining the Class Precedence List}
\label{classprecedence}

No changes.

%%%------------------------------------------------------------------------------
\subsubsection{Generic Functions and Methods}
\label{generic}

As mentioned earlier, the {\em functional interface} in \CLOS\ is intended 
to enable runtime creation and changes to classes, generic functions and 
methods. It does not belong to \clos0. Hence, only the extended top level forms 
{\tt defgeneric} and {\tt defmethod} can be used in order to create generic 
functions or to define methods. Moreover, \clos0\ does not support anonymous 
global generic functions.

Consequently, {\tt add-method}, {\tt remove-method}, {\tt with-added-methods} 
as well as the macro {\tt generic-function} do not belong to \clos0. 
Neither {\tt with-added-methods} not the special forms {\tt generic-flet} 
and {\tt generic-labels} do belong to the ANSI \CL\ proposal.
The present definition of \clos0\ does not support local generic functions, 
either. According to the results of the ANSI \CL\ discussion, {\tt 
generic-flet} and {\tt generic-labels} might be added to future versions of 
\clos0.

The {\tt defgeneric} and {\tt defmethod} defining forms do not support the 
options {\tt :generic-function-class} and {\tt :method-class}.

With respect to \cl0\ modules, there is one more restriction which has to hold: If methods 
for one generic function {\it f} are to be defined within two different 
modules, the appropriate generic function definition for {\it f}, either 
explicitly or implicitely using {\tt defmethod}, has to occur within one 
common imported module.

%%------------------------------------------------------------------------------
\subsubsection{Method Selection and Combination}
\label{methodselection}

Only the standard method combination and the {\em simple built in} method 
combination types are available in \clos0. The macro {\tt 
define-method-combination} is not part of \clos0. Hence, the {\tt 
:method-combination} option to {\tt defgeneric}, if supplied, is restricted 
to the constant keys {\tt :+}, {\tt :and}, {\tt :append}, {\tt :list}, {\tt 
:max}, {\tt :min}, {\tt :nconc}, {\tt :or}, {\tt :or}, and, of course, {\tt 
:standard}, which is the default. Although method combinations will not 
become {\em meta objects}, the macro {\tt define-method-combination} might 
become part of future versions of \clos0.

Calls to {\tt call-next-method} must not specify arguments. In \clos0\ the 
set of next methods will be determined statically. If {\tt 
call-next-method} were called with incompatible arguments, the chain of 
next methods could be broken and hence could change at runtime. This is a 
wellknown problem in \CLOS\ which the definition of \clos0\ intends to 
avoid.

The generic functions {\tt no-applicable-method} and {\tt 
no-next-method} will signal an uncontinuable error and hence will abort the 
program execution. (Cf. section \ref{classes}.)

The functions {\tt invalid-method-error} and {\tt method-combination-error} 
as well as the macro {\tt call-method} do not belong to \clos0.

%%------------------------------------------------------------------------------
\subsubsection{Meta-objects}
\label{metaobjects}

\clos0\ does not support any of the {\em meta objects} built into \CLOS, 
nor does it support {\em metaclasses}. Therefore, creating meta objects is 
impossible in \clos0. It is an error to apply {\tt class-of} to classes, generic 
functions, or methods. The function {\tt class-of} is restricted only to be 
defined on instances of user defined classes. Moreover, method 
combinations are not objects, and, therefore, do not belong to a class 
either. 

%%------------------------------------------------------------------------------
\subsubsection{Object Creation and Initialization}
\label{objects}

In \clos0, the function {\tt make-instance} is not generic. It is restricted 
only to be defined on class objects, which, in turn, only can be created 
using {\tt find-class} applied to a constant symbol which has to denote a defined 
named class. {\tt (make-instance '<{\it name}\/> ...)} is a syntactical 
abbreviation for {\tt (make-instance (find-class '<{\it name}\/>) ...)}.

The generic functions {\tt shared-initialize} and {\tt allocate-instance} 
do not belong to \clos0. Thus, \clos0\ defines the standard behavior as 
specified for \CLOS\ to be the semantics of {\tt make-instance}. The one 
and only exception is that {\tt initialize-instance} might be customized 
using specialized method definitions. With respect to \cl0\ modules, one 
more restriction has to hold: If a {\tt (defclass {\it A} ...)} defining 
form occurs within one module, all of the method definitions specializing 
{\tt initialize-instance} over {\it A}\/ have to be defined within the same 
module.

Although {\tt reinitialize-instance} does belong to \clos0, it will not be 
called implicitely. The reason is that neither class redefinitions nor 
changing the class on an instance, i.e.\ {\tt change-class}, are allowed.

The generic functions {\tt update-\-instance-\-for-\-redefined-\-class} and {\tt 
update-\.instance-\.for-\.different-\.class} do not belong to \clos0.

%%------------------------------------------------------------------------------
\subsubsection{Redefining Classes}
\label{classredefine}

We classify class redefinitions to belong to the program development 
interface and, thus, they are rejected in \clos0. The complete metaobject 
protocol defined in this section of \cite{Steele90} does not belong to 
\clos0. Class redefinition will not be supported in future versions of 
\clos0, whereas some parts of the metaobject protocol, like {\tt 
shared-initialize}, might become part of future versions.

%%------------------------------------------------------------------------------
\subsubsection{Changing the Class of an Instance}
\label{changeclass}

The function {\tt change-class} and the complete metaobject protocol 
defined here are not, and will not be supported by \clos0. The reason is 
that we want to ensure the correctness of type inference results, which, of 
course, would be impossible with respect to objects eventually changing 
their classes.

One possible way in order to weaken this restriction is to allow a {\em 
monotonic} version of {\tt change-class}, i.e.\ changing the class of an 
instance only to more specific classes. But even this restricted version 
would disable many optimizations which rely on the concrete knowledge of 
the direct class of an instance. Hence, we decided to drop {\tt 
change-class} at all.

%%------------------------------------------------------------------------------
\subsubsection{Reinitializing an Instance}
\label{reinitialize}

Since {\tt shared-initialize} does not belong to \clos0, the generic 
function {\tt reinitialize-instance} does not call {\tt  
shared-initialize}.

%------------------------------------------------------------------------------
\subsection{Functions in the Programmer Interface}
\label{functions}

The following two lists are intended to summarize those operators, which 
are, and those, which are not part of \clos0. 

\begin{itemize}

\item \code{call-next-method} \\ 
Restricted. Cf. section \ref{methodselection} for details.
%Um die M"oglichkeit der Inlinekompilation in
%einem Modul zu schaffen k"onnte man an dieser Stelle noch die Monotonie
%der \cpl s auf den Teilb"aumen des Klassengraphen fordern.

\item \code{class-of} \\ 
Restricted. Cf. section \ref{metaobjects} for details.

\item \code{class-name}

\item \code{defclass} \\ 
Restricted. Cf. section \ref{classes} for details.

\item \code{defgeneric} \\ 
Restricted. Cf. section \ref{generic} and section \ref{methodselection} for details.

\item \code{defmethod} \\ 
Restricted. Cf. section \ref{generic} for details.

\item \code{find-class} \\ 
Restricted. Cf. section \ref{classes} and section \ref{objects} for details.

\item \code{initialize-instance} \\ 
Restricted. Cf. section \ref{objects} for details.

\item \code{make-instance} \\ 
Restricted. Cf. section \ref{objects} for details.

\item \code{next-method-p}

\item \code{no-applicable-method} \\
Restricted. Cf. section \ref{methodselection} for details.

\item \code{no-next-method} \\
Restricted. Cf. section \ref{methodselection} for details.

\item \code{print-object}

\item \code{reinitialize-instance} \\ 
Restricted. Cf. section \ref{reinitialize} for details.

\item \code{slot-boundp} 

\item \code{slot-makunbound}  

\item \code{slot-missing} 
Restricted. Cf. section \ref{classes} for details.

\item \code{slot-unbound}
Restricted. Cf. section \ref{classes} for details.

\item \code{slot-value} 

\item \code{with-accessors}

\item \code{with-slots}

\end{itemize}

The following list summarizes those operators which do not belong to 
\clos0:

\begin{itemize} 
\item[$-$] \code{add-method}
\item[$-$] \code{call-method}  
\item[$-$] \code{change-class}
\item[$-$] \code{compute-applicable-methods}
\item[$-$] \code{define-method-combination}
\item[$-$] \code{documentation}
\item[$-$] \code{ensure-generic-function}
\item[$-$] \code{find-method}
\item[$-$] \code{function-keywords}
\item[$-$] \code{generic-flet}
\item[$-$] \code{generic-funtion}
\item[$-$] \code{generic-labels}
\item[$-$] \code{invalid-method-error}
\item[$-$] \code{make-instances-obsolete}
\item[$-$] \code{method-combination-error}
\item[$-$] \code{remove-method}
\item[$-$] \code{shared-initialize}
\item[$-$] \code{slot-exists-p}
\item[$-$] \code{update-instance-for-different-class}
\item[$-$] \code{update-instance-for-redefined-class}
\item[$-$] \code{with-added-methods}
\end{itemize}


% Local Variables: 
% mode: latex
% TeX-master: "cl0-engl"
% End: 

%% To be added.

\section{Acknowledgements}

We would like to thank Harry Bretthauer and J\"urgen Kopp for an intensive 
discussion and many suggestions helping us to improve the definition of 
\clos0.

\nocite{APPLY/GMD/XIII/1}
\nocite{Knutzen91}
\nocite{Burkart91}
\bibliographystyle{named} \bibliography{apply,own,other}

\end{document}

