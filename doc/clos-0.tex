% Projekt  : APPLY - A Practicable And Portable Lisp Implementation
%            ------------------------------------------------------
%            CLOS_0 Sprachbeschreibung fuer cl0-engl.tex
% 
% $Revision: 1.1 $
% $Id: clos-0.tex,v 1.1 1993/09/03 15:01:38 wg Exp $

Like \cl0, which essentially is defined in order to exclude {\em
reflection} from \CL, \clos0\ will be defined as a subset of \CLOS,
which essentially excludes unrestricted uses of reflection.  The
reason again is, that we want to ensure the quality of {\em global}
program analysis and optimization for \cl0\ application programs which
use object oriented programming.

But, unlike reflection in \CL, in \CLOS\ the notion of reflection has
various aspects. First of all, the distinction between program and
data is not as simple and wellknown for the object system as for the
non object oriented kernel of \CL. Instances are data, of course. But
should classes, i.e.\ types, be program or data? Should classes be
instances of metaclasses?  At present, there are many different
answers to questions like this. The intention of \clos0\ is to call
those operations, which work on classes or generic functions as data
objects, {\em reflective}.  Hence, they are excluded from the
language. On the other hand, metaobject programming, at least in a
restricted manner, seems to be very useful even for application
programming. At least some parts of the metaobject protocol should be
supported by \clos0, whereas other parts, like class redefinition,
definitely belong to the development environment.  Although some
reflection should be part of \clos0, at present it is not. We
have to be very careful in improving the present definition in this
aspect, because we have to ensure both, the strict subset property,
and the quality of static program analysis.

The strategy we will use to improve \clos0\ in order to include
metaobject programming will be to keep those operations apart from
runtime, like runtime macro expansion has been removed from  the
modern \Lisp\ language versions. 

The definition on \clos0\ included here has to be taken as a
first attempt in order to define a suitable subset of \CLOS. At
present, the restrictions are very rigorous in order to achieve our
goals: There is no meta object protocol. Classes, generic functions
and methods are program, not data. There is no possibility to define
or redefine classes, methods, or generic functions at runtime,
nor to add methods to generic functions. Changing an
instance's class is impossible. There are objects representing classes,
but they have no class. It is an error to access a class's
class. Moreover, \clos0\ ensures class objects only to represent
classes which are statically defined within the program text.

%%% Internally they are represented as instances of the one and only
%%% invisible and illegal metaclass {\tt standard-class}.

Again, not every useful \CL\ $+$ \CLOS\ program belongs and will
belong to the subset defined within this document. But we hope that it
will be rich enough to express the most requirements of \Lisp\
application programs.

%%------------------------------------------------------------------------------
\subsection{Programmer Interface Concepts}
\label{concepts}

As before, we will follow the section headings of the original
specification, i.e.\ the chapter 28 of \cite{Steele90}, in order to
define \clos0\ by stating changes and differences with respect to the
original.  Section \ref{functions} then will summarize both the
functionality that is supported by \clos0\, together with the
restrictions defined, and those functions, macros, or generic
functions which are not.

Restrictions are defined both for the {\em metaobject protocol} and
the {\em functional interface}, which \CLOS\ provides in order to define
classes, generic functions, and methods at runtime. Therefore, in \clos0\ 
the only legal method to create classes, generic functions or methods is to 
use {\tt defclass}, {\tt defgeneric}, or {\tt defmethod}, which are {\em 
defining forms} and hence are restricted to occur as extended toplevel 
forms only. (cf.\ section \ref{progstructure})

The generic functions which are classified to be part of the
metaobject protocol, are:

\begin{itemize}
\item[$\bullet$] {\tt shared-initialize}
\item[$\bullet$] {\tt update-instance-for-different-class}
\item[$\bullet$] {\tt update-instance-for-redefined-class}
\item[$\bullet$] {\tt compute-applicable-methods}
\item[$\bullet$] {\tt ensure-generic-function}
\item[$\bullet$] {\tt compute-effective-method}
\item[$\bullet$] {\tt allocate-instance}
\item[$\bullet$] {\tt make-instances-obsolete}
\end{itemize}

They do not belong to \clos0. 
As a consequence, many functions and generic functions which remain in
\clos0, like {\tt initialize-instance},
actually do behave as this is defined for the standard
situation, but their behavior is not customizable using method
definitions for generic functions of the metaobject protocol. 

We classify {\tt add-method}, {\tt remove-method}, {\tt generic-function}, and calls to 
{\em make-instance} with metaclasses
as class arguments to belong to the functional interface. \clos0\ does not 
support them. Moreover, \clos0\ does not support metaclasses at all.
Thus, there is no way to create instances of metaclasses, like anonymous classes or generic 
functions, nor to dynamically change or enlarge the set of methods which 
belong to a generic function.

The lists above are intended to give a first impression of the 
nature of 
the restrictions defined within this document. Hence, they are not complete, 
i.e.\ there are more important restrictions, and more functions, like {\tt 
change-class}, are excluded, although they are neither part of the metaobject 
protocol nor do belong to the functional interface.
We refer to the subsequent sections for details.

%%------------------------------------------------------------------------------
\subsubsection{Classes}
\label{classes}

The classes {\tt standard-class}, {\tt standard-generic-function}, and {\tt 
standard-method} are metaclasses in \CLOS. Therefore, they do not belong to \clos0. 

Unlike in \CLOS, {\tt find-class} is restricted to be defined on constant
symbols which name user defined classes, i.e.\ a classes which are defined 
using {\tt defclass}.

The function {\tt make-instance} is not generic. It is restricted to be 
defined only on class objects, which, in turn, only can be accessed using 
{\tt find-class}. Hence, \clos0\ ensures class objects only to represent 
classes which are defined using the extended toplevel form {\tt 
defclass} and which, therefore, are statically known from the program text. 
In order to avoid unnecassary syntactical transformations, {\tt 
(make-instance '<{\it name}\/> ...)} internally is transformed to {\tt 
(make-instance (find-class '<{\it name}\/>) ...)}, and hence, is legal in 
\clos0.

The {\tt defclass} macro does not support the class option {\tt 
:metaclass}.

Moreover, the {\tt :allocation} slot option must not override the {\tt 
:allocation} option of an inherited slot. 

The standard methods defined for {\tt slot-unbound} and {\tt 
slot-missing} will signal and uncontinuable error and hence will abort the 
program execution. Note, that, according to the current definition of 
\clos0, there is no way to define methods for {\tt slot-unbound} and {\tt 
slot-missing} because they would have to be specialized over metaclasses. 
This is true for {\tt no-applicable-method} and {\tt no-next-method} as 
well (cf. section \ref{methodselection}). Hence, an implementation can choose to 
call simple error functions instead of these generic functions.

%%------------------------------------------------------------------------------
\subsubsection{Inheritance}
\label{inheritance}

The {\tt :allocation} slot option is restricted not to override specified {\tt 
:allocation} options of inherited slots. Thus, unlike in \CLOS, shared 
slots cannot be shadowed in \clos0.

%%------------------------------------------------------------------------------
\subsubsection{Integrating Types and Classes}
\label{classtypes}

The functions {\tt typep}, {\tt type-of}, and {\tt subtypep} are restricted 
according to the sections \ref{predicates} and \ref{type} of the 
\cl0\ part of this document. As 
mentioned there already, these very rigorous restrictions will be weakened 
with respect to classes in future versions of \cl0\ + \clos0.

The metaclasses {\tt built-in-class} and {\tt structure-class} do not 
belong to \clos0.

%%------------------------------------------------------------------------------
\subsubsection{Determining the Class Precedence List}
\label{classprecedence}

No changes.

%%%------------------------------------------------------------------------------
\subsubsection{Generic Functions and Methods}
\label{generic}

As mentioned earlier, the {\em functional interface} in \CLOS\ is intended 
to enable runtime creation and changes to classes, generic functions and 
methods. It does not belong to \clos0. Hence, only the extended top level forms 
{\tt defgeneric} and {\tt defmethod} can be used in order to create generic 
functions or to define methods. Moreover, \clos0\ does not support anonymous 
global generic functions.

Consequently, {\tt add-method}, {\tt remove-method}, {\tt with-added-methods} 
as well as the macro {\tt generic-function} do not belong to \clos0. 
Neither {\tt with-added-methods} not the special forms {\tt generic-flet} 
and {\tt generic-labels} do belong to the ANSI \CL\ proposal.
The present definition of \clos0\ does not support local generic functions, 
either. According to the results of the ANSI \CL\ discussion, {\tt 
generic-flet} and {\tt generic-labels} might be added to future versions of 
\clos0.

The {\tt defgeneric} and {\tt defmethod} defining forms do not support the 
options {\tt :generic-function-class} and {\tt :method-class}.

With respect to \cl0\ modules, there is one more restriction which has to hold: If methods 
for one generic function {\it f} are to be defined within two different 
modules, the appropriate generic function definition for {\it f}, either 
explicitly or implicitely using {\tt defmethod}, has to occur within one 
common imported module.

%%------------------------------------------------------------------------------
\subsubsection{Method Selection and Combination}
\label{methodselection}

Only the standard method combination and the {\em simple built in} method 
combination types are available in \clos0. The macro {\tt 
define-method-combination} is not part of \clos0. Hence, the {\tt 
:method-combination} option to {\tt defgeneric}, if supplied, is restricted 
to the constant keys {\tt :+}, {\tt :and}, {\tt :append}, {\tt :list}, {\tt 
:max}, {\tt :min}, {\tt :nconc}, {\tt :or}, {\tt :or}, and, of course, {\tt 
:standard}, which is the default. Although method combinations will not 
become {\em meta objects}, the macro {\tt define-method-combination} might 
become part of future versions of \clos0.

Calls to {\tt call-next-method} must not specify arguments. In \clos0\ the 
set of next methods will be determined statically. If {\tt 
call-next-method} were called with incompatible arguments, the chain of 
next methods could be broken and hence could change at runtime. This is a 
wellknown problem in \CLOS\ which the definition of \clos0\ intends to 
avoid.

The generic functions {\tt no-applicable-method} and {\tt 
no-next-method} will signal an uncontinuable error and hence will abort the 
program execution. (Cf. section \ref{classes}.)

The functions {\tt invalid-method-error} and {\tt method-combination-error} 
as well as the macro {\tt call-method} do not belong to \clos0.

%%------------------------------------------------------------------------------
\subsubsection{Meta-objects}
\label{metaobjects}

\clos0\ does not support any of the {\em meta objects} built into \CLOS, 
nor does it support {\em metaclasses}. Therefore, creating meta objects is 
impossible in \clos0. It is an error to apply {\tt class-of} to classes, generic 
functions, or methods. The function {\tt class-of} is restricted only to be 
defined on instances of user defined classes. Moreover, method 
combinations are not objects, and, therefore, do not belong to a class 
either. 

%%------------------------------------------------------------------------------
\subsubsection{Object Creation and Initialization}
\label{objects}

In \clos0, the function {\tt make-instance} is not generic. It is restricted 
only to be defined on class objects, which, in turn, only can be created 
using {\tt find-class} applied to a constant symbol which has to denote a defined 
named class. {\tt (make-instance '<{\it name}\/> ...)} is a syntactical 
abbreviation for {\tt (make-instance (find-class '<{\it name}\/>) ...)}.

The generic functions {\tt shared-initialize} and {\tt allocate-instance} 
do not belong to \clos0. Thus, \clos0\ defines the standard behavior as 
specified for \CLOS\ to be the semantics of {\tt make-instance}. The one 
and only exception is that {\tt initialize-instance} might be customized 
using specialized method definitions. With respect to \cl0\ modules, one 
more restriction has to hold: If a {\tt (defclass {\it A} ...)} defining 
form occurs within one module, all of the method definitions specializing 
{\tt initialize-instance} over {\it A}\/ have to be defined within the same 
module.

Although {\tt reinitialize-instance} does belong to \clos0, it will not be 
called implicitely. The reason is that neither class redefinitions nor 
changing the class on an instance, i.e.\ {\tt change-class}, are allowed.

The generic functions {\tt update-\-instance-\-for-\-redefined-\-class} and {\tt 
update-\.instance-\.for-\.different-\.class} do not belong to \clos0.

%%------------------------------------------------------------------------------
\subsubsection{Redefining Classes}
\label{classredefine}

We classify class redefinitions to belong to the program development 
interface and, thus, they are rejected in \clos0. The complete metaobject 
protocol defined in this section of \cite{Steele90} does not belong to 
\clos0. Class redefinition will not be supported in future versions of 
\clos0, whereas some parts of the metaobject protocol, like {\tt 
shared-initialize}, might become part of future versions.

%%------------------------------------------------------------------------------
\subsubsection{Changing the Class of an Instance}
\label{changeclass}

The function {\tt change-class} and the complete metaobject protocol 
defined here are not, and will not be supported by \clos0. The reason is 
that we want to ensure the correctness of type inference results, which, of 
course, would be impossible with respect to objects eventually changing 
their classes.

One possible way in order to weaken this restriction is to allow a {\em 
monotonic} version of {\tt change-class}, i.e.\ changing the class of an 
instance only to more specific classes. But even this restricted version 
would disable many optimizations which rely on the concrete knowledge of 
the direct class of an instance. Hence, we decided to drop {\tt 
change-class} at all.

%%------------------------------------------------------------------------------
\subsubsection{Reinitializing an Instance}
\label{reinitialize}

Since {\tt shared-initialize} does not belong to \clos0, the generic 
function {\tt reinitialize-instance} does not call {\tt  
shared-initialize}.

%------------------------------------------------------------------------------
\subsection{Functions in the Programmer Interface}
\label{functions}

The following two lists are intended to summarize those operators, which 
are, and those, which are not part of \clos0. 

\begin{itemize}

\item \code{call-next-method} \\ 
Restricted. Cf. section \ref{methodselection} for details.
%Um die M"oglichkeit der Inlinekompilation in
%einem Modul zu schaffen k"onnte man an dieser Stelle noch die Monotonie
%der \cpl s auf den Teilb"aumen des Klassengraphen fordern.

\item \code{class-of} \\ 
Restricted. Cf. section \ref{metaobjects} for details.

\item \code{class-name}

\item \code{defclass} \\ 
Restricted. Cf. section \ref{classes} for details.

\item \code{defgeneric} \\ 
Restricted. Cf. section \ref{generic} and section \ref{methodselection} for details.

\item \code{defmethod} \\ 
Restricted. Cf. section \ref{generic} for details.

\item \code{find-class} \\ 
Restricted. Cf. section \ref{classes} and section \ref{objects} for details.

\item \code{initialize-instance} \\ 
Restricted. Cf. section \ref{objects} for details.

\item \code{make-instance} \\ 
Restricted. Cf. section \ref{objects} for details.

\item \code{next-method-p}

\item \code{no-applicable-method} \\
Restricted. Cf. section \ref{methodselection} for details.

\item \code{no-next-method} \\
Restricted. Cf. section \ref{methodselection} for details.

\item \code{print-object}

\item \code{reinitialize-instance} \\ 
Restricted. Cf. section \ref{reinitialize} for details.

\item \code{slot-boundp} 

\item \code{slot-makunbound}  

\item \code{slot-missing} 
Restricted. Cf. section \ref{classes} for details.

\item \code{slot-unbound}
Restricted. Cf. section \ref{classes} for details.

\item \code{slot-value} 

\item \code{with-accessors}

\item \code{with-slots}

\end{itemize}

The following list summarizes those operators which do not belong to 
\clos0:

\begin{itemize} 
\item[$-$] \code{add-method}
\item[$-$] \code{call-method}  
\item[$-$] \code{change-class}
\item[$-$] \code{compute-applicable-methods}
\item[$-$] \code{define-method-combination}
\item[$-$] \code{documentation}
\item[$-$] \code{ensure-generic-function}
\item[$-$] \code{find-method}
\item[$-$] \code{function-keywords}
\item[$-$] \code{generic-flet}
\item[$-$] \code{generic-funtion}
\item[$-$] \code{generic-labels}
\item[$-$] \code{invalid-method-error}
\item[$-$] \code{make-instances-obsolete}
\item[$-$] \code{method-combination-error}
\item[$-$] \code{remove-method}
\item[$-$] \code{shared-initialize}
\item[$-$] \code{slot-exists-p}
\item[$-$] \code{update-instance-for-different-class}
\item[$-$] \code{update-instance-for-redefined-class}
\item[$-$] \code{with-added-methods}
\end{itemize}


% Local Variables: 
% mode: latex
% TeX-master: "cl0-engl"
% End: 
